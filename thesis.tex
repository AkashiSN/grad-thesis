% ------------------------------------------------------
% preamble
% ------------------------------------------------------

\documentclass[autodetect-engine, dvipdfmx-if-dvi, a4paper, ja=standard, 10pt, textwidth-limit=45]{bxjsbook}

\usepackage{ifthen}
\usepackage{ifxetex,ifluatex,ifuptex}

\ifthenelse{\boolean{xetex}}{
  % XeLaTeXの時
}{\ifthenelse{\boolean{luatex}}{
  % LuaLaTeXの時
  \usepackage{luatexja}
}{\ifthenelse{\boolean{upTeX}}{
  % upLaTeXの時
}{%else (pLaTeXと仮定する)
  % pLaTeX の時
}}}

\renewcommand{\figurename}{Fig}
\usepackage{graphicx}
\usepackage{url}

\begin{document}

% ------------------------------------------------------
% front cover
% ------------------------------------------------------

{\LARGE
\vspace{1.0cm}
\begin{center}
  令和元年度卒業研究論文
\end{center}}

{\huge
\vspace{1.0cm}
\begin{center}
  URLの情報指向型クラシフィケーション
\end{center}}

{\LARGE
\vspace{13cm}
\begin{center}
  2020年2月7日(金)
\end{center}
\vspace{0.2cm}
\begin{center}
  \begin{tabular}{ccc}
    指導教員 & 井上一成 & 教授
  \end{tabular}
\end{center}
\vspace{0.2cm}
\begin{center}
  明石工業高等専門学校 \\
  電気情報工学科
\end{center}
\begin{center}
  \begin{tabular}{crl}
  報告者 & E1533 & 西 総一朗
  \end{tabular}
\end{center}}

\thispagestyle{empty}

\clearpage

% ------------------------------------------------------
% table of contents
% ------------------------------------------------------

\pagenumbering{roman}
\tableofcontents

\clearpage

% ------------------------------------------------------
% body
% ------------------------------------------------------

\pagenumbering{arabic}
\setcounter{page}{1}

\chapter{序論}
\section{TCP/IPの課題}
1983年から今日のインターネットと呼ばれているネットワークにおいて通信プロトコルTCP/IPがデファクトスタンダードとなった\cite{Brief_History_of_the_Internet}.
約20年前のインターネットのトラヒックや利用形態は現在とは大きく異なっている.
1992年の全世界のインターネットトラフィックは1日あたり約100 GBであったが,その10年後の2002年には1秒あたり100 GBに増え,2017年には1秒あたり45,000 GB以上に到達した.
また利用形態も2017年においてはトラヒックの75\%をビデオコンテンツが占めている.
Ciscoによると全世界のインターネットトラヒックは2022年には150,700 GB/秒となりその82\%をビデオコンテンツが占めると予測されている\cite{Cisco_Visual_Networking_Index}.

また,インターネットの使用目的も変遷している.
当初はインターネットを高性能コンピュタあるいは高性能プリンタを利用するように,様々なリソースを遠隔から共有することが主な目的であった.
現在は情報の共有,情報の取得といった情報のやり取りが中心となっている.
それに伴って,通信形態も変化している.
従来のTCP/IPはホスト中心のHost-to-Hostの通信形態であり,IPプロトコルは位置情報であるネットワークアドレスを用いてホストアドレスを指定するというロケーション・オリエンテッド\footnote{Location-oriented: 地理的指向な}な通信であった.
ところが,現在は情報をユーザに送るというインフォメーション・セントリック\footnote{Information-Centric: 情報指向な}な通信形態に変わりつつある.

このようにTCP/IPの通信形態と現在のインターネットに求められている通信形態との間の差が広がっている.
情報の効率的な取得のために
P2P\footnote{Peer to Peer: インターネットにおいて一般的に用いられるクライアント・サーバ型モデルでは,データを保持・提供するサーバとそれに対してデータを要求・アクセスするクライアントという2つの立場が固定されているのに対して,各ピアに対して対等 にデータの提供及び要求・アクセスを行う自立分散型のネットワークモデル}や
CDN\footnote{Content Delivery Network: 頻繁に使われるWebサイトがあると一つのノード(サーバ)だけでは耐えきれないのでいくつかのノードにデータを分散しておき,各ユーザは分散したノードに接続して情報を取得するという方法}などの新しいプロトコルが提案された.
しかし,これらはロケーション・セントリックなTCP/IPネットワーク上のプロトコルであるので本質的な解決ではない.
本来,情報を取得するという行為に対して,ネットワークアドレスやホストアドレスなどを意識する必要はなく,
もし近くにある通信機器が当該コンテンツ(情報)\footnote{参考文献\cite{Networking_Named_Content}では情報(Information)とコンテンツ(Contents)は同様の意味で用いられている.本稿でも同様の意味で用いる.}を持っておりそこから情報を取得できるなら,それはより効率的であり将来の通信量増大にも対応できると考えられる.
そこで,情報を効率的に取得するために情報指向ネットワーク:\ Information-Centric-Network (ICN)\cite{Networking_Named_Content}というプロトコル体系が提案された\cite{ICN_prototype}.

\section{情報指向ネットワーク}
情報指向ネットワーク(ICN)においてユーザはサーバのIPアドレスではなくコンテンツ名を指定してコンテンツ取得要求を行い,
そのコンテンツ要求を受け取った近隣のルータやノードが当該コンテンツを保持していた場合,それらはユーザに対してそのコンテンツを直接転送するプロトコル体系である.
ICNへのアプローチとして様々なが研究がなされているが,現在最も多くの研究者により研究されているNamed Data Networking (NDN)\cite{NDN} 及びその前身であるContent Centric Networking (CCN)\cite{CCN} を代表的なICNアーキテクチャとして述べる.
CCNアーキテクチャはパロアルト研究所\footnote{Palo Alto Research Center (PARC) : アメリカ合衆国のカリフォルニア州パロアルトにある研究開発企業}により研究されているICNの先駆となった本格的なアーキテクチャである.
また,US Future Internet Architectureプログラム\footnote{NSF FUTURE INTERNET ARCHITECTURE PROJECT (http://www.nets-fia.net/)}によって資金提供されたNDNプロジェクトは,CCNアーキテクチャをさらに発展させたものである.

NDNにおけるコンテンツ名の命名規則は階層構造になっており,現在のインターネットで流通している識別子であるUniform-Resource-Locator (URL)に似ている.
たとえば,コンテンツ名は\url{/aueb.gr/ai/main.html}となる.
ただし,コンテンツ名は必ずしもURLとは一致せず,最初のセクション\footnote{"/"で区切られた部分をセクションと呼ぶ}はDNS名またはIPアドレスなどの人間が読める形式である必要もない.
代わりに,NDNでは各名前セクションは,ドットで区切られた人間が読み取れる文字列やハッシュ値など,何でもかまいません.

\begin{figure}[h]
  \center
  \includegraphics[width=140mm]{fig/CCN_NDN_Routing-crop.pdf}
  \caption{The CCN/NDN architecture.CR stands for Content Router, FIB for Forwarding Information Base, PIT for Pending Interest Table, CS for Content Store (Excerpt from \cite{A_Survey_of_Information-Centric_Networking_Research}).}
  \label{fig:ccn_ndn_routing}
\end{figure}

\section{本研究の目的}

\chapter{シミュレーションプログラム}
\section{プログラムの概要}

\chapter{衝突数の検証}
\section{ハッシュのみを用いたとき}
\section{URLの分類手法を利用するとき}
\section{ハッシュとURLの分類手法を併用したとき}

\bibliography{thesis}
\bibliographystyle{junsrt}

\end{document}
