% ------------------------------------------------------
% preamble
% ------------------------------------------------------

\documentclass[autodetect-engine, dvipdfmx-if-dvi, a4paper, ja=standard, 10pt]{bxjsbook}

\usepackage{ifthen}
\usepackage{ifxetex,ifluatex,ifuptex}

\ifthenelse{\boolean{xetex}}{
  % XeLaTeXの時
}{\ifthenelse{\boolean{luatex}}{
  % LuaLaTeXの時
  \usepackage{luatexja}
}{\ifthenelse{\boolean{upTeX}}{
  % upLaTeXの時
}{%else (pLaTeXと仮定する)
  % pLaTeX の時
}}}

\usepackage{url}

\begin{document}

% ------------------------------------------------------
% front cover
% ------------------------------------------------------

{\LARGE
\vspace{1.0cm}
\begin{center}
  令和元年度卒業研究論文
\end{center}}

{\huge
\vspace{1.0cm}
\begin{center}
  URLの情報指向型クラシフィケーション
\end{center}}

{\LARGE
\vspace{13cm}
\begin{center}
  2020年2月7日(金)
\end{center}
\vspace{0.2cm}
\begin{center}
  \begin{tabular}{ccc}
    指導教員 & 井上一成 & 教授
  \end{tabular}
\end{center}
\vspace{0.2cm}
\begin{center}
  明石工業高等専門学校 \\
  電気情報工学科
\end{center}
\begin{center}
  \begin{tabular}{crl}
  報告者 & E1533 & 西 総一朗
  \end{tabular}
\end{center}}

\thispagestyle{empty}

\clearpage

% ------------------------------------------------------
% table of contents
% ------------------------------------------------------

\pagenumbering{roman}
\tableofcontents

\clearpage

% ------------------------------------------------------
% body
% ------------------------------------------------------

\pagenumbering{arabic}
\setcounter{page}{1}

\chapter{序論}
\section{TCP/IPの課題}
1983年から今日のインターネットと呼ばれているネットワークにおいて通信プロトコルTCP/IPがデファクトスタンダードとなった\cite{Brief_History_of_the_Internet}.
約20年前のインターネットのトラヒックや利用形態は現在とは大きく異なっている.
1992年の全世界のインターネットトラフィックは1日あたり約100 GB,その10年後の2002年には1秒あたり100 GBに増え, 2017年には1秒あたり45,000 GB以上に到達した.
また利用形態も2017年においてはトラヒックの75\%をビデオコンテンツが占めている.
Ciscoによると全世界のインターネットトラヒックは2022年には150,700 GB/秒となりその82\%をビデオコンテンツが占めると予測されている\cite{Cisco_Visual_Networking_Index}.

また,インターネットの使用目的も変遷している.
当初はインターネットを高性能コンピュタあるいは高性能プリンタを利用するように,様々なリソースを遠隔から共有することが主な目的であった.
現在は情報の共有,情報の取得といった情報のやり取りが中心となっている.
それに伴って,通信形態も変化している.
従来のTCP/IPはホスト中心のHost-to-Hostの通信形態であり,IPプロトコルは位置情報であるネットワークアドレスを用いてホストアドレスを指定するというロケーション・セントリックな通信であった.
しかし,現在はInformation-to-User指向で情報をユーザに送るという通信形態に変わりつつある.
このようにTCP/IPの通信形態と現在のインターネットに求められている通信形態との間の差が広がっている.
この差を解消するためにP2P(Peer to Peer)やCDN(Content Delivery Network)などの新しいプロトコルが提案された.
しかし,これらはロケーション・セントリックなTCP/IPネットワーク上のプロトコルであるので本質的な解決ではない.


\section{情報指向形ネットワーク}
\section{本研究の目的}

\chapter{シミュレーションプログラム}
\section{プログラムの概要}

\chapter{衝突数の検証}
\section{ハッシュのみを用いたとき}
\section{URLの分類手法を利用するとき}
\section{ハッシュとURLの分類手法を併用したとき}

\bibliography{thesis}
\bibliographystyle{junsrt}

\end{document}
