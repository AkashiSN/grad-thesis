\usepackage{ifthen}
\usepackage{ifxetex,ifluatex,ifuptex}

\ifthenelse{\boolean{xetex}}{
  % XeLaTeXの時
}{\ifthenelse{\boolean{luatex}}{
  % LuaLaTeXの時
  \usepackage{luatexja}
  \usepackage{geometry}
}{\ifthenelse{\boolean{upTeX}}{
  % upLaTeXの時
  \usepackage[dvipdfmx]{graphicx}
}{%else (pLaTeXと仮定する)
  % pLaTeX の時
}}}

\usepackage{booktabs}
\usepackage{fancyvrb}
\usepackage{here}
\usepackage{url}

\setpagelayout*{top=2.5truecm, bottom=2.5truecm, left=2truecm, right=2truecm}

%% Hyphenation setting
\hyphenpenalty=1000\relax
\exhyphenpenalty=1000\relax
\sloppy

\pagestyle{empty}

% 行数, カラム間の幅指定
\setlength{\columnsep}{5mm}
\setlength{\textheight}{46\baselineskip}

% 図と本文との余白
\setlength\textfloatsep{5pt}
\setlength\intextsep{5pt}

\makeatletter
\renewcommand{\@maketitle}{
  \begin{center}
    {\LARGE \@title \par}
    \vskip 1.5em
    {\large \lineskip .5em
      \begin{tabular}[t]{c}
        \@author
      \end{tabular}
      \par
    }
  \end{center}
  \par
  \vskip 1.5em
}

\renewcommand\section{\@startsection {section}{1}%
{\z@}%   インデント大きさ(長さを指定)
{0.8ex}% 見出しの前のスペース(長さを指定)
{0.8ex}% 見出しの後のスペース(長さを指定)
{\normalfont\Large\bfseries}}% 見出しの書式

\renewcommand\subsection{\@startsection {subsection}{1}%
{\z@}%   インデント大きさ(長さを指定)
{0.8ex}% 見出しの前のスペース(長さを指定)
{0.8ex}% 見出しの後のスペース(長さを指定)
{\normalfont\large\bfseries}}% 見出しの書式

\renewcommand\subsubsection{\@startsection {subsubsection}{1}%
{\z@}%   インデント大きさ(長さを指定)
{0.8ex}% 見出しの前のスペース(長さを指定)
{0.8ex}% 見出しの後のスペース(長さを指定)
{\normalfont\normalsize\bfseries}}% 見出しの書式

\renewcommand{\thefootnote}{*\arabic{footnote}}

\let\oldenumerate\enumerate
\renewcommand{\enumerate}{
  \oldenumerate
  \setlength{\itemsep}{0pt}
  \setlength{\parskip}{0pt}
}

\let\olditemize\itemize
\renewcommand{\itemize}{
  \olditemize
  \setlength{\itemsep}{0pt}
  \setlength{\parskip}{0pt}
}

\let\oldtable\table
\let\endoldtable\endtable
\renewenvironment{table}{\oldtable}{
  \centering
  \endoldtable
}

\makeatother
